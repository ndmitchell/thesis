%include thesis.fmt

\chapter{Function Index}

\todo{Give the definition site for all functions and types defined in the thesis. Give types of Prelude functions at least, and ideally some implementations.}

\todo{Things to watch out for: symbols appearing correct, items such as propAnd which render as /\ but are textually different, things like ||> which require additional escaping}


\printindex


\section{Library Functions}

\begin{comment}
\begin{code}
import Prelude(Char,Show,Eq(..),Int,Num(..),Ord(..),otherwise,repeat,Read,Bool(..))
data IO a = IO a
instance Monad IO
\end{code}
\end{comment}


The following functions come from the following libraries:

\ind{Prelude}\ind{String}\ind{flip}\ind{id}\ind{error}\ind{undefined}
\ind{True}\ind{False}\ind{Bool}\ind{fst}\ind{snd}\ind{const}\ind{not}
\ind{read}\ind{reads}\ind{show}
\begin{code}
module Prelude where

type String = [Char]

{-" |data Bool = False || True| "-}

not :: Bool -> Bool
not x = if x then False else True

(&&),(||) :: Bool -> Bool -> Bool
a && b = if a then b else True
a || b = if a then True else b

(.) :: (b -> c) -> (a -> b) -> a -> c
(.) f g x = f (g x)

flip :: (a -> b -> c) -> b -> a -> c
flip f a b = f b a

id :: alpha -> alpha
id x = x

-- terminate with an error
error :: String -> alpha

undefined :: alpha
undefined = error "undefined"

fst (x,y) = x
snd (x,y) = y

const                   :: a -> b -> a
const x _               =  x

-- show an item as a string
show :: Show a => a -> String

-- read an item from a string
read :: Read a => String -> a

-- read an item, returning all possible parses
reads :: Read a => String -> [(a, String)]
\end{code}

\ind{List}\ind{map}\ind{any}\ind{all}\ind{zipWith}\ind{elem}\ind{length}\ind{iterate}\ind{zip}
\ind{notElem}\ind{splitAt}\ind{take}\ind{drop}\ind{break}\ind{span}\ind{takeWhile}\ind{dropWhile}
\ind{lookup}\ind{null}\ind{tail}\ind{head}\ind{replicate}\ind{filter}\ind{concat}\ind{nub}\ind{foldr}
\ind{and}\ind{or}\ind{concatMap}\ind{reverse}\ind{partition}
\begin{code}
module Data.List where

{-" |data [alpha] = [] || (:) alpha [alpha]| "-}

length :: [alpha] -> Int
length []      = 0
length (x:xs)  = 1 + length xs

map :: (alpha -> beta) -> [alpha] -> [beta]
map f []      = []
map f (x:xs)  = f x : map f xs

foldr            :: (a -> b -> b) -> b -> [a] -> b
foldr _ z []     =  z
foldr f z (x:xs) =  f x (foldr f z xs)

(++) :: [alpha] -> [alpha] -> [alpha]
xs ++ ys = foldr (:) ys xs

or,and :: [Bool] -> Bool
or                      =  foldr (||) False
and                     =  foldr (&&) True

any, all :: (alpha -> Bool) -> [alpha] -> Bool
any  f = or   . map f
all  f = and  . map f

null                    :: [a] -> Bool
null []                 =  True
null (_:_)              =  False

head                    :: [a] -> a
head (x:_)              =  x
head []                 =  error "head of empty list"

tail                    :: [a] -> [a]
tail (_:xs)             =  xs
tail []                 =  error "tail of empty list"

elem,notElem :: Eq alpha => alpha -> [alpha] -> Bool
elem x = any (== x)
notElem x = all (/= x)

zipWith :: (alpha -> beta -> gamma) -> [alpha] -> [beta] -> [gamma]
zipWith f (x:xs) (y:ys) = f x y : zipWith f xs ys
zipWith f _ _ = []

zip :: [a] -> [b] -> [(a,b)]
zip = zipWith (,)

lookup :: (Eq a) => a -> [(a,b)] -> Maybe b
lookup key []           =  Nothing
lookup key ((x,y):xys)  | key == x   = Just y
                        | otherwise  = lookup key xys

iterate :: (a -> a) -> a -> [a]
iterate f x =  x : iterate f (f x)

splitAt                :: Int -> [a] -> ([a],[a])
splitAt n xs           =  (take n xs, drop n xs)

take :: Int -> [a] -> [a]
take n _      | n <= 0 =  []
take _ []              =  []
take n (x:xs)          =  x : take (n-1) xs

drop :: Int -> [a] -> [a]
drop n xs     | n <= 0 =  xs
drop _ []              =  []
drop n (_:xs)          =  drop (n-1) xs

span, break :: (a -> Bool) -> [a] -> ([a],[a])
span p xs = (takeWhile p xs, dropWhile p xs)
break p                 =  span (not . p)

dropWhile               :: (a -> Bool) -> [a] -> [a]
dropWhile _ []          =  []
dropWhile p xs@(x:xs')
            | p x       =  dropWhile p xs'
            | otherwise =  xs

takeWhile               :: (a -> Bool) -> [a] -> [a]
takeWhile _ []          =  []
takeWhile p (x:xs) 
            | p x       =  x : takeWhile p xs
            | otherwise =  []

replicate               :: Int -> a -> [a]
replicate n x           =  take n (repeat x)

filter :: (a -> Bool) -> [a] -> [a]
filter _pred []    = []
filter pred (x:xs)
  | pred x         = x : filter pred xs
  | otherwise	   = filter pred xs

concat :: [[a]] -> [a]
concat = foldr (++) []

concatMap               :: (a -> [b]) -> [a] -> [b]
concatMap f             =  foldr ((++) . f) []

nub                     :: (Eq a) => [a] -> [a]
nub []             =  []
nub (x:xs)         =  x : nub (filter (/= x) xs)

reverse :: [a] -> [a]
reverse l =  rev l []
  where  rev []     a = a
         rev (x:xs) a = rev xs (x:a)
         
partition :: (a -> Bool) -> [a] -> ([a], [a])
partition p xs = (filter p xs, filter (not . p) xs)
\end{code}

\ind{isSpace}\ind{Char}
\begin{code}
module Data.Char where

-- is a character a space
isSpace :: Char -> Bool
\end{code}


\ind{Maybe}\ind{Nothing}\ind{Just}\ind{maybe}\ind{fromJust}\ind{fromMaybe}\ind{isNothing}
\begin{code}
module Data.Maybe where

data Maybe alpha = Nothing | Just alpha

maybe :: beta -> (alpha -> beta) -> Maybe alpha -> beta
maybe nothing just Nothing   = nothing
maybe nothing just (Just x)  = just x

fromMaybe :: a -> Maybe a -> a
fromMaybe x = maybe x id

fromJust          :: Maybe a -> a
fromJust Nothing  = error "Maybe.fromJust: Nothing"
fromJust (Just x) = x

isNothing         :: Maybe a -> Bool
isNothing Nothing = True
isNothing _       = False
\end{code}

\ind{Monad}\ind{return}\ind{liftM}\ind{mapM}\ind{mapM_}\ind{sequence}\ind{sequence_}
\ind{Functor}\ind{fmap}
\begin{code}
module Control.Monad where

class Monad m where
		(>>=) :: m alpha -> (alpha -> m beta) -> m beta
		(>>) :: m alpha -> m beta -> m beta
		return :: alpha -> m alpha

class Functor f where
	  fmap :: (a -> b) -> f a -> f b

(=<<) :: Monad m => (a -> m b) -> m a -> m b
(=<<) = flip (>>=)

liftM :: Monad m => (alpha -> beta) -> m alpha -> m beta
liftM f x = x >>= (return . f)

mapM            :: Monad m => (a -> m b) -> [a] -> m [b]
mapM f as       =  sequence (map f as)

mapM_           :: Monad m => (a -> m b) -> [a] -> m ()
mapM_ f as      =  sequence_ (map f as)

sequence       :: Monad m => [m a] -> m [a] 
sequence ms = foldr k (return []) ms
	where k m ms = m >>= \x -> ms >>= \xs -> return (x:xs)

sequence_        :: Monad m => [m a] -> m () 
sequence_ ms     =  foldr (>>) (return ()) ms
\end{code}

\ind{State}\ind{runState}\ind{get}\ind{put}\ind{evalState}
\begin{code}
module Control.Monad.State where

newtype State s alpha = State {runState :: s -> (alpha, s)}
instance Monad (State s)

evalState :: State s alpha -> s -> alpha
evalState s = fst . runState s

get :: State s s
put :: s -> State s ()
\end{code}

\ind{Function}\ind{on}
\begin{code}
module Data.Function where

on :: (b -> b -> c) -> (a -> b) -> a -> a -> c
on g f x y = g (f x) (f y)
\end{code}

\ind{System}\ind{IO}\ind{putChar}\ind{putStr}\ind{putStrLn}\ind{print}\ind{getArgs}\ind{getContents}
\begin{code}
module System.IO where

-- write out a character to the output
putChar :: Char -> IO ()

putStr :: String -> IO ()
putStr = mapM_ putChar

putStrLn :: String -> IO ()
putStrLn x = putStr x >> putChar '\n'

print :: Show alpha => alpha -> IO ()
print = putStrLn . show

-- get the command line arguments
getArgs :: IO [String]

-- get the input stream
getContents :: IO String
\end{code}

