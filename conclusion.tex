%include paper.fmt

\chapter{Conclusions and Future Work}


\section{Uniplate}

We have presented the Uniplate library. It defines the classes |Uniplate| and |Biplate|, along with a small set of operations to perform queries and transformations. We have illustrated by example that the boilerplate required in our system is less than in others (\S\ref{sec:results_boilerplate}), and that we can achieve these results without sacrificing speed (\S\ref{sec:results_speed}). Our library is both practical and portable, finding use in a number of applications, and using fewer extensions to the Haskell language than alternatives.

The restriction to a uniformly typed value set in a traversal allows the power of well-developed techniques for list processing such as list-comprehensions to be exploited. We feel this decision plays to Haskell's strengths, without being limiting in practice.

There is scope for further speed improvements: for example, use of continuation passing style may eliminate tuple construction and consumption, and list fusion may be able to eliminate some of the intermediate lists in |uniplate|. We have made extensive practical use of the Uniplate library, but there may be other traversals which deserve to be added.

The use of boilerplate reduction strategies in Haskell is not yet ubiquitous, as we feel it should be. We have focused on simplicity throughout our design, working within the natural typed design of Haskell, rather than trying to extend it. Hopefully the removal of complicated language features (particularly `scary' types) will allow a wider base of users to enjoy the benefits of boilerplate-free programming.

Other people have implemented Uniplate

The ideas behind the Uniplate library have been used extensively, in projects including the Yhc compiler \citep{me:yhc_core}, the Catch tool \citep{me:catch_tfp}, the Reach tool \cite{naylor:reach} and the Reduceron \cite{naylor:reduceron}. In Catch there are over 100 Uniplate traversals. We think there is more scope to use the existing Uniplate library.



\section{Supero}

Our supercompiler is simple -- the Core transformation is expressed in just 300 lines of Haskell. Yet it replicates many of the performance enhancements of GHC in a more general way. We have modified some of the techniques from supercompilation, particularly with respect to let bindings and generalisation. Our initial results are promising, but incomplete. Using our supercompiler in conjunction with GHC we obtain an average runtime improvement of 16\% for the imaginary section of the nofib suite. To quote Simon Peyton Jones, ``an average runtime improvement of 10\%, against the baseline of an already well-optimised compiler, is an excellent result'' \cite{spj:specconstr}.

There are three main areas for future work:

\begin{description}
\item[More Benchmarks] The fifteen benchmarks presented in this chapter are not enough. We would like to obtain results for larger programs, including all the remaining benchmarks in the nofib suite.
\item[Runtime Performance] Earlier versions of Supero \cite{me:supero_ifl} managed to obtain substantial speed ups on benchmarks such as exp3\_8. The Bernouilli benchmark is currently problematic. There is still scope for improvement.
\item[Compilation Speed] The compilation times are tolerable for benchmarking and a final optimised release, but not for general use. Basic profiling shows that over 90\% of supercompilation time is spent testing for a homeomorphic embedding, which is currently done in a na\"{i}ve manner -- dramatic speedups should be possible.
\end{description}

The Programming Language Shootout\footnote{\url{http://shootout.alioth.debian.org/}} has shown that low-level Haskell can compete with low-level imperative languages such as C. Our goal is that Haskell programs can be written in a high-level declarative style, yet still perform competitively.


\section{Firstify}

Higher-order functions are very useful, but may pose difficulties for certain types of analysis. Using the method we have described, it is possible to remove most functional values from most programs. A user can still write higher-order programs, but an analysis tool can work on equivalent first-order programs.

Our method has already found practical use within the Catch tool, and we hope it can be of benefit to others. We have released our tool, both as a command line program, and as a library that analysis programs can invoke. Currently the implementation is based around the core language from the Yhc compiler, which restricts our input programs to the Haskell 98 language. By making use of the GHC front end, we could deal with many language extensions.

Our method is whole program, requiring sources for all function definitions. This requirement both increases transformation time, and precludes the use of closed source libraries. We may be able to relax this requirement, precomputing first-order variants of libraries, or permitting some components of the program to be ignored.

We have developed our method for analysis, not performance. However, for many simple examples, the resultant program performs better than the original. By restricting rules that reduce sharing, our defunctionalisation method may be appropriate for integration into an optimising compiler.

The use of a numeric termination bound in the homeomorphic embedding is regrettable, but practically motivated. We need further research to determine if such a numeric bound is necessary, or if other measures could be used.

Many analysis methods, in fields such as strictness analysis and termination analysis, start out first-order and are gradually extended to work in a higher-order language. Defunctionalisation offers an alternative approach, instead of extending the analysis method, we transform the functional values away, enabling more analysis methods to work on a greater range of programs.


\section{Catch}

We have described the design, implementation and application of Catch,
an analysis tool for safe pattern-matching in Haskell 98.  Two key
design decisions in Catch simplify the analysis and make it scalable:
(1) the target of analysis is a very small, first-order core language;
(2) there are finitely many value-set-defining constraints per type.
Decision (1) requires a translation from the full language that avoids
the introduction of analysis bottlenecks such as a mini-interpreter.
Decision (2) inevitably limits the expressive power of
constraints; yet it does not prevent the expression of uniform recursive
constraints on the deep structure of values, as in MP-constraints.

Practical evaluation, using Catch to analyse widely distributed examples
in Haskell 98, confirms our claim to give results for programs of moderate
size written in the full language. But it does also reveal a frequent need
to modify programs, widening definitions or narrowing applications,
before Catch can verify pattern-match safety.

Outcomes of example applications could drive the exploration of more
powerful variants of MP-constraints, with a greater (but still finite)
number of expressible constraints per type.  More demanding tests of
scalability could include the application of Catch to a Haskell compiler,
or indeed to Catch itself.

A tool such as Catch might do more harm than good if it sometimes wrongly
declares that a program cannot fail.  We claim that the Catch analysis
is sound, but we have not formally proved soundness with respect to a
suitable semantics for the core language.

Like many researchers, we are interested in narrowing the gap between the
exactness of constructive mathematics and the scalability of practical
programming systems.  We hope that Catch or its successors can provide
a small but useful bridge crossing part of that gap.
