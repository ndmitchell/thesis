%include paper.fmt

\chapter{Conclusion}


\section{Uniplate}


\section{Conclusions and Future Work}
\label{sec:conclusion}

We have presented the Uniplate library. It defines the classes |Uniplate| and |Biplate|, along with a small set of operations to perform queries and transformations. We have illustrated by example that the boilerplate required in our system is less than in others (\S\ref{sec:results_boilerplate}), and that we can achieve these results without sacrificing speed (\S\ref{sec:results_speed}). Our library is both practical and portable, finding use in a number of applications, and using fewer extensions to the Haskell language than alternatives.

The restriction to a uniformly typed value set in a traversal allows the power of well-developed techniques for list processing such as list-comprehensions to be exploited. We feel this decision plays to Haskell's strengths, without being limiting in practice.

There is scope for further speed improvements: for example, use of continuation passing style may eliminate tuple construction and consumption, and list fusion may be able to eliminate some of the intermediate lists in |uniplate|. We have made extensive practical use of the Uniplate library, but there may be other traversals which deserve to be added.

The use of boilerplate reduction strategies in Haskell is not yet ubiquitous, as we feel it should be. We have focused on simplicity throughout our design, working within the natural typed design of Haskell, rather than trying to extend it. Hopefully the removal of complicated language features (particularly `scary' types) will allow a wider base of users to enjoy the benefits of boilerplate-free programming.



\section{Supero}

Our supercompiler is simple -- the Core transformation is expressed in just 300 lines of Haskell. Yet it replicates many of the performance enhancements of GHC in a more general way. We have modified some of the techniques from supercompilation, particularly with respect to let bindings and generalisation. Our initial results are promising, but incomplete. Using our supercompiler in conjunction with GHC we obtain an average runtime improvement of 16\% for the imaginary section of the nofib suite. To quote Simon Peyton Jones, ``an average runtime improvement of 10\%, against the baseline of an already well-optimised compiler, is an excellent result'' \cite{spj:specconstr}.

There are three main areas for future work:

\begin{description}
\item[More Benchmarks] The fifteen benchmarks presented in this chapter are not enough. We would like to obtain results for larger programs, including all the remaining benchmarks in the nofib suite.
\item[Runtime Performance] Earlier versions of Supero \cite{me:supero_ifl} managed to obtain substantial speed ups on benchmarks such as exp3\_8. The Bernouilli benchmark is currently problematic. There is still scope for improvement.
\item[Compilation Speed] The compilation times are tolerable for benchmarking and a final optimised release, but not for general use. Basic profiling shows that over 90\% of supercompilation time is spent testing for a homeomorphic embedding, which is currently done in a na\"{i}ve manner -- dramatic speedups should be possible.
\end{description}

The Programming Language Shootout\footnote{\url{http://shootout.alioth.debian.org/}} has shown that low-level Haskell can compete with low-level imperative languages such as C. Our goal is that Haskell programs can be written in a high-level declarative style, yet still perform competitively.

