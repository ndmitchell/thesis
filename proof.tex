%include thesis.fmt

%format <| = "\mathbin{\lhd}"
%format |> = "\mathbin{\rhd}"
%format -< = "\mbox{\ensuremath{<\hspace{-1em}-}}"
%format :- = "\!\rightsquigarrow{}\!"
%format :* = "\star{}"
%format :| = "\ddagger{}"
%format ^. = "\!\cdot{}\!"
%format ^* = "^\ast{}\!"
%format ^ = "\;\hat{}\;"
%format undefined   = "\Varid{undefined}"
%format `mergeVal` = "\sqcap{}"
%format mergeVal = "(\sqcap{})"
%format && = "\mathrel{\hbox{\textsf{\& \hspace{-1em} \&}}}"
%format || = "\mathrel{||}"

%format `propAnd` = "\;\ensuremath{\wedge}\;"
%format `propOr` = "\;\ensuremath{\vee}\;"
%format propAnd = "(\ensuremath{\wedge})"
%format propOr = "(\ensuremath{\vee})"
%format propAnds = "\mathsf{andP}"
%format propOrs = "\mathsf{orP}"
%format propLit = "\mathsf{lit}"
%format propMap = "\mathsf{mapP}"
%format propTrue = "\mathsf{true}"
%format propFalse = "\mathsf{false}"
%format propBool = "\mathsf{bool}"
%format propIsTrue = "\mathsf{isTrue}"
%format propIsFalse = "\mathsf{isFalse}"
%format `propImp` = "\ensuremath{\Longrightarrow}"
%format propImp = "(\ensuremath{\Longrightarrow})"
%format ==> = "\ensuremath{\Rightarrow}"

%format pre' = pre "^{\prime}"

%format `Sat` = "\backtick{Sat}"

\newcommand{\rec}[1]{\hspace{-0.75ex}_{#1}}
%format Rec0 = "\rec{\mathrm{0}}"
%format RecN = "\rec{\Varid{n}}"
%format RecN1 = "\rec{\Varid{n}\text{\tiny{+}}\mathrm{1}}"


\newcommand{\para}[1]{\vspace{2mm}\noindent\textbf{#1}}


\chapter{Pattern Match Analysis Proof}

\section{Preliminaries}

\begin{code}
type Selector  =  (CtorName, Int)
type ArgPos    =  Int

var    :: VarName   -> Either ArgPos (Expr, Selector)
isRec  :: Selector  -> Bool
\end{code}

\begin{code}
data Prop alpha

propAnd, propOr           :: Prop alpha -> Prop alpha -> Prop alpha
propAnds, propOrs         :: [Prop alpha] -> Prop alpha
propMap                   :: (alpha -> Prop beta) -> Prop alpha -> Prop beta
propTrue, propFalse       :: Prop alpha
propBool                  :: Bool -> Prop alpha
propLit                   :: alpha -> Prop alpha
\end{code}

\begin{code}
data Sat alpha = Sat alpha Constraint

(-<)  :: alpha -> [CtorName] -> Prop (Sat alpha)
(|>)  :: Selector -> Constraint -> Constraint
(<|)  :: CtorName -> Constraint -> Prop (Sat ArgPos)
\end{code}

\begin{code}
precond   :: FuncName -> Prop (Sat ArgPos)
prePost   :: FuncName -> Constraint -> Prop (Sat ArgPos)
reduce    :: Prop (Sat Expr) -> Prop (Sat ArgPos)

substP ::  [alpha] -> Prop (Sat ArgPos) -> Prop (Sat alpha)
substP xs p = propMap (\(Sat i k ) -> propLit $ Sat (xs!!i) k) p
\end{code}

\begin{code}
data Constraint  =  Any
                 |  Con CtorName [Constraint]
\end{code}

\begin{code}
pre ::  Expr -> Prop (Sat Expr)
pre (EVar v       ) = propTrue
pre (ECon c xs_   ) = propAnds (map pre xs_)
pre (EFun f xs_   ) = pre' f xs `propAnd` propAnds (map pre xs_)
    where pre' f xs = substP xs (precond f)
pre (ECase x as_  ) = pre x `propAnd` propAnds (map alt as_)
    where alt (EAlt c vs_ y) = on -< (ctors c \\ [c]) `propOr` pre y
\end{code}

\ignore\begin{code}
precond  :: FuncName -> Prop (Sat ArgPos)
precond Rec0   f =  if f == "error" then propFalse else propTrue
precond RecN1  f =  precond RecN f `propAnd` reduce (pre{precond RecN} (body f))
\end{code}

\begin{code}
reduce :: Prop (Sat Expr) -> Prop (Sat ArgPos)
reduce = propMap (\(Sat x k) -> red x k)

red :: Expr -> Constraint -> Prop (Sat ArgPos)
red (EVar   v     )  k =  case var v of
                            Left   i       -> propLit (i `Sat` k)
                            Right  (on,c)  -> red on (c |> k)
red (ECon c xs_   )  k =  reduce $ substP xs_ (c <| k)
red (EFun f xs_   )  k =  reduce $ substP xs_ (prePost f k)
red (ECase x as_  )  k =  propAnds (map alt as_)
    where alt (EAlt c vs_ y) = reduce (on -< (ctors c \\ [c])) `propOr` red y k
\end{code}

\ignore\begin{code}
prePost :: FuncName -> Constraint -> Prop (Sat ArgPos)
prePost Rec0   f k =  propTrue
prePost RecN1  f k =  prePost RecN f k `propAnd` reduce{prePost RecN} (propLit $ body f `Sat` k)
\end{code}



\section{Argument for Correctness}
\label{secC:correct}

\newcommand{\theorem}[2]{#2  \hspace{5mm} (\textit{#1}) \\}
\newcommand{\proof}[1]{\paragraph{(\textit{#1})}}
%format subst a b c = c "[" a "\backslash" b "]"

We first show that the algorithm is correct given certain conditions on the constraint language, then that each of the three constraint languages satisfy the necessary conditions.

\subsection{Correctness of Algorithm}

The algorithm is correct if given a function application whose arguments meet the precondition, it will not call |error|. The proof is structured as an evaluator with an explicit |Bottom| within it -- if this case is encountered the program crashes. We use equational reasoning upon the evaluator to show that the |bottom| case can be eliminated by introducing a check on the first precondition. This corresponds to a proof of Figures \ref{figC:property_fixp}, \ref{figC:precondition}, \ref{figC:precond_fixp} and \ref{figC:prop}.

The idea is that as we transform code we leave impossible cases out, which means we can reason about which things to ignore. We insert |bottom| in the impossible cases, then remove it later, as necessary.

\subsubsection{Helper Functions}

First we introduce functions and definitions which will be required later.

\begin{code}
data Val  =  Val CtorName [Val]
          |  Bottom

isBottom, notBottom :: Expr -> Bool
isBottom Bottom = True
isBottom (Val c xs) = any isBottom xs
notBottom = not . isBottom

reduce' :: FuncName -> Prop (Sat VarName)
reduce' f = substP (args f) $ reduce (body f)

precond' :: FuncName -> Prop (Sat VarName)
precond' f = substP (args f) $ precond f

sat :: Prop (Sat Expr) -> Bool
\end{code}

We define the following helper functions.

We start by outlining an evaluator:

\begin{code}
data Expr = Make CtorName [Expr]
          | Call FuncName [Expr]
          | Var  VarName
          | Case Expr [Alt]

data Alt = Alt CtorName [VarName] Expr

eval :: Expr -> Val
eval (Make c xs_   ) = Val c (map eval xs_)
eval (Call f xs_)  | f == "error" = Bottom
                   | otherwise = eval $ (body f)[args f / xs_]
eval (Case x as_  ) = case eval x of
    Val c xs_ -> head [y [vs_ / xs_] | Alt c' vs_ y <- as_, c == c']
    Bottom -> Bottom
\end{code}

We use |subst vs_ xs_| to denote replacing the free variables |vs_| with |xs_|.

There is no |eval| alternative for |Var|, as an expression at evaluation will not have any free variables, as |subst| will remove them.

\subsubsection{Theorems}

We wish to prove:

\theorem{pre/eval}
{|sat (pre x) ==> notBottom (eval x)|}

To do this, we will need to prove the following theorems:

\theorem{sat/and}
{|sat (x `propAnd` y) = sat x && sat y|}


%\begin{code}
%(sat/all)       forall (xs :: [Sat Expr]) . sat (andP xs) == all (sat xs)
%(evalPre/evalSafe)    pre x ==> (evalPre x == evalSafe x)
%(error/precond)  f == "error" ==> not (satPrecond "error" xs)
%
%(satPrecond/sat)
%satPrecond f xs
%sat $ substP xs $ reduce (pre (body f))
%
%(subst/subst')
%substP xs y == subst' (zip vs xs) (substP vs y)
%-- provided length xs == length vs and all vs are otherwise not used
%
%reduce' = reduce but not changing var's to Int's
%subst' = subst but over a predicate
%
%(reduce/reduce')
%substP xs (reduce x) == subst' (zip vs xs) (reduce' x)
%where vs are the free variables in vs
%
%(sat/reduce)
%sat $ reduce x => sat x
%
%(sat/subst)
%sat x && all sat ys => sat $ subst (zip vs ys) x
%
%
%sat $ subst' sub
%\end{code}


\subsubsection{The Proof}

We prove each lemma in turn.

\proof{pre/eval}

By induction on possible expressions:




The axiom (evalPre/eval) says that |evalPre| returns either the same result as |eval|, or a |Bottom| containing result. Therefore the condition (pre/evalPre) implies (pre/eval).


\paragraph{Defining |evalPre|.} We first define |evalPre| and |evalSafe|:

It is clear from the definition that (evalPre/eval) is true, as only additional |Bottom| results are produced. The |bottom| inserted into |evalSafe| is safe as it will never be evaluated. The intention is that |evalSafe| represents a call which does not produce |Bottom|. By transforming |evalSafe| we can show this, which proves the (pre/evalPre) lemma. While transforming |evalSafe| we can change any result from.

\paragraph{Transforming |evalSafe|, |Make| equation.} We start by performing a pattern match on the expression passed to |evalSafe|, and looking at each equation in turn.

\begin{code}
evalSafe x@(Make c xs) | sat (pre x) = eval x
    -- inline eval
evalSafe x@(Make c xs) | sat (andP (map pre xs)) = Val c (map eval xs_)
    -- (evalPre/eval)
evalSafe x@(Make c xs) | sat (andP (map pre xs)) = Val c (map evalPre xs_)
    -- (sat/all)
evalSafe x@(Make c xs) | all (sat . pre) xs = Val c (map evalPre xs)
    -- (evalPre/evalSafe)
evalSafe x@(Make c xs) | all (sat . pre) xs = Val c (map evalSafe xs)
\end{code}

\paragraph{Transforming |evalSafe|, |Call| equation.}

This is the |Call| equation.

\begin{code}
evalSafe x@(Call f xs) | sat (pre x) = eval x
    -- where introduction
evalSafe x@(Call f xs) | sat (pre x) = eval x
    where  arg = args f
           sub = zip args xs
           bod = body f
    -- inline pre
evalSafe (Call f xs) | satPrecond f xs && all (sat xs) = eval x
    -- inline transform with negation to the inverse
evalSafe (Call f xs) | not $ satPrecond f xs && all (sat . pre) xs = bottom
                     | otherwise = eval x
    -- inline eval
evalSafe (Call f xs) | not $ satPrecond f xs && all (sat . pre) xs = bottom
                     | f == "error" = Bottom
                     | otherwise = eval $ subst sub bod
    -- (error/satPrecond)
evalSafe (Call f xs) | not $ satPrecond f xs && all (sat . pre) xs = bottom
                     | otherwise = eval $ subst sub bod
    -- split the |bottom| case
evalSafe (Call f xs) | not $ satPrecond f xs = bottom
                     | not $ all (sat . pre) xs = bottom
                     | otherwise = eval $ subst sub bod
    -- (satPrecond/sat)
evalSafe (Call f xs) | not $ sat $ substP xs $ reduce (pre bod) = bottom
                     | not $ all (sat . pre) xs = bottom
                     | otherwise = eval $ subst sub bod
    -- (reduce/reduce')
evalSafe (Call f xs) | not $ sat $ subst' sub $ reduce' (pre bod) = bottom
                     | not $ all (sat . pre) xs = bottom
                     | otherwise = eval $ subst sub bod
    -- (sat/subst)
evalSafe (Call f xs) | not $ sat $ reduce' (pre bod) = bottom
                     | not $ all (sat . pre) xs = bottom
                     | otherwise = eval $ subst sub bod
    -- (sat/reduce)
evalSafe (Call f xs) | not $ sat $ pre bod = bottom
                     | not $ all (sat . pre) xs = bottom
                     | otherwise = eval $ subst sub bod
    -- (sat/subst)
evalSafe (Call f xs) | not $ sat $ subst sub $ pre bod = Bottom
                     | not $ all (sat . pre) xs = bottom
                     | otherwise = eval $ subst sub bod

-- sat/reduce => sat
evalSafe (Call f xs) | not $ sat $ subst' s $ pre $ y = Bottom
                     ....

-- subst/pre => pre/subst
evalSafe (Call f xs) | not $ sat $ pre $ subst s y = Bottom
                     ....

-- now can call evalSafe
\end{code}

subst/pre probably requires:

\begin{code}
(subst' xs $ pre $ body x) && all sat xs => pre $ subst xs $ body x
\end{code}


We know that |precond "error" == false|

as the initial value of |precond "error"| is |false|, and the values only get more restrictive, and |false| is the most restrictive value. Therefore |f == "error" ==> wrong (precond "error")|. We can now strengthen the |Call| alternative with:

\todo{The case for case}

\todo{Now prove sat/reduce => sat lemma}


\begin{comment}
\begin{code}
preEval x | reduce (pre x) == false = bottom
          | otherwise = eval x
\end{code}

Obviously this code is less defined than |eval|, and produces |bottom| in more places, so is a safe restriction.

Now we can case expand |preEval|:

\begin{code}
preEval x@(Make c xs_) | reduce (pre x) == false = bottom
                     | otherwise = eval x
\end{code}

\begin{code}
preEval (Make c xs_) | reduce (andP (map pre xs_)) == false = bottom
                     | otherwise = Val c (map eval xs_)
\end{code}

If we assume that |reduce x ==> x|, [is that true?] given no free variables, then we can rewrite to:

\begin{code}
preEval (Make c xs_) = Val c (map preEval xs_)
\end{code}

Now we can deal with the other cases:



Finally, we prove that is a function meets its precondition, it is satisfied.





Which we can now see is an implication



\begin{code}
eval (Make c xs_   ) = Val c (map eval xs_)
eval (Case x as_  ) = head [subst y (zip vs_ xs_) | Alt c' vs_ y <- as_, c == c']
    where Val c xs_ = eval x
eval (Call f xs_)   | not $ satisfiesP $ subst (precond f) xs_ = bottom
                   | otherwise = subst (zip (args f) xs_) (body f)
\end{code}
\end{comment}



\subsection{Correctness of BP-constraints}


\subsection{Correctness of RE-constraints}


\subsection{Correctness of MP-constraints}

